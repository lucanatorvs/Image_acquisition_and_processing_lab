\section {Assignment 5 \\ {Convolution}}
\label {sec:assignment_5}

We wrote a convolutie 5x5 kernel with all 1's. The kernel put over the image from lab 2 is shown in figure \ref{fig:5x5}. Note how the output of the filter will slightly blur the image.

\begin{figure}[h!]
    \centering
    \includegraphics[width=0.45\textwidth]{5x5.png}
    \caption{5x5 kernel over image from lab 2}
    \label{fig:5x5}
\end{figure}

The code we used to make the kernel is shown in listing \ref{lst:5x5}. and the full code can be found in appendix \ref{sec:appendix_C}.

We slightly deviated from the original assignment function prototype by replacing the int* argument to a 2D array pointer. We think that is not possible to easily call the kernel by reference without changing this argument, and makes the function more sensible to use. The size is fixed at compile time however.

The code does not make use of any OpenCV iteration templates, which would have drastically improved the performance. However, the assignemnt stated that it was not allowed to rely on OpenCV functionality despite having to use the cv2::Mat class.

The code below will loop over the height and width of the image, and on each pixel loop through the kernel passed as an argument. The result is the use of nested for loops, which quadratically increases processing time. It is highly inefficient and took roughly 13 seconds to execute on my machine (5587.52 BogoMips)

\begin{lstlisting}[language=C, caption=5x5 kernel, label=lst:5x5]
    //int* from original assignment replaced with pointer to the kernel instead, because that makes more sense
    void convolve5 (Mat* inputImg, Mat* outImg, int (*kernel5)[5][5]) { 
        //I assume the mat is CV_8UC1 since I want to process BGR channels individually
        Size s = inputImg -> size(); // get size of image
        const uint8_t kernel_size = 5; // todo: replace with sizeof
        uint8_t out[s.height][s.width]; // create output array
        int x ,y, h, w, i, j, sum; // declare variables
        
        std::cout << "Input type was : " << inputImg->type() << std::endl; //debug
      
        for (h=0; h<s.height; h++) { // for each row
            for(w=0; w<s.width; w++){ // for each column
                sum = 0; // reset sum
                for(i=0 ;i<kernel_size; i++ ){ // for each kernel row
                    for(j=0; j<kernel_size; j++){ // for each kernel column
                        y=h-i+1; x=w-j+1; // calculate the position of the pixel in the image
                        // if the pixel is outside the image, set it to the border
                        if(y<0) y=0;
                        if(y>s.height-2) y=s.height-2; 
                        if(x<0) x=0;
                        if(x>s.width-2) x=s.width-2;
                        sum += ((*kernel5)[i][j]) * inputImg->at<uint8_t>(y,x); // add the product of the kernel and the pixel to the sum
                    }
                }
                sum /= kernel_size*kernel_size; //divide the result of the pixel by 5^2
                if(sum<0) sum=0; // if the result is negative, set it to 0
                if(sum>255) sum=255; // if the result is greater than 255, set it to 255
                std::memcpy(outImg->data, out, s.height*s.width*sizeof(uint8_t)); // copy the result to the output array
                out[h][w]=(uint8_t)sum; // set the result to the output array
            }
        }
    }
    
\end{lstlisting}
